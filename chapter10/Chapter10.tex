
% Default to the notebook output style

    


% Inherit from the specified cell style.




    
\documentclass[11pt]{article}

    
    
    \usepackage[T1]{fontenc}
    % Nicer default font than Computer Modern for most use cases
    \usepackage{palatino}

    % Basic figure setup, for now with no caption control since it's done
    % automatically by Pandoc (which extracts ![](path) syntax from Markdown).
    \usepackage{graphicx}
    % We will generate all images so they have a width \maxwidth. This means
    % that they will get their normal width if they fit onto the page, but
    % are scaled down if they would overflow the margins.
    \makeatletter
    \def\maxwidth{\ifdim\Gin@nat@width>\linewidth\linewidth
    \else\Gin@nat@width\fi}
    \makeatother
    \let\Oldincludegraphics\includegraphics
    % Set max figure width to be 80% of text width, for now hardcoded.
    \renewcommand{\includegraphics}[1]{\Oldincludegraphics[width=.8\maxwidth]{#1}}
    % Ensure that by default, figures have no caption (until we provide a
    % proper Figure object with a Caption API and a way to capture that
    % in the conversion process - todo).
    \usepackage{caption}
    \DeclareCaptionLabelFormat{nolabel}{}
    \captionsetup{labelformat=nolabel}

    \usepackage{adjustbox} % Used to constrain images to a maximum size 
    \usepackage{xcolor} % Allow colors to be defined
    \usepackage{enumerate} % Needed for markdown enumerations to work
    \usepackage{geometry} % Used to adjust the document margins
    \usepackage{amsmath} % Equations
    \usepackage{amssymb} % Equations
    \usepackage{textcomp} % defines textquotesingle
    % Hack from http://tex.stackexchange.com/a/47451/13684:
    \AtBeginDocument{%
        \def\PYZsq{\textquotesingle}% Upright quotes in Pygmentized code
    }
    \usepackage{upquote} % Upright quotes for verbatim code
    \usepackage{eurosym} % defines \euro
    \usepackage[mathletters]{ucs} % Extended unicode (utf-8) support
    \usepackage[utf8x]{inputenc} % Allow utf-8 characters in the tex document
    \usepackage{fancyvrb} % verbatim replacement that allows latex
    \usepackage{grffile} % extends the file name processing of package graphics 
                         % to support a larger range 
    % The hyperref package gives us a pdf with properly built
    % internal navigation ('pdf bookmarks' for the table of contents,
    % internal cross-reference links, web links for URLs, etc.)
    \usepackage{hyperref}
    \usepackage{longtable} % longtable support required by pandoc >1.10
    \usepackage{booktabs}  % table support for pandoc > 1.12.2
    \usepackage[normalem]{ulem} % ulem is needed to support strikethroughs (\sout)
                                % normalem makes italics be italics, not underlines
    

    
    
    % Colors for the hyperref package
    \definecolor{urlcolor}{rgb}{0,.145,.698}
    \definecolor{linkcolor}{rgb}{.71,0.21,0.01}
    \definecolor{citecolor}{rgb}{.12,.54,.11}

    % ANSI colors
    \definecolor{ansi-black}{HTML}{3E424D}
    \definecolor{ansi-black-intense}{HTML}{282C36}
    \definecolor{ansi-red}{HTML}{E75C58}
    \definecolor{ansi-red-intense}{HTML}{B22B31}
    \definecolor{ansi-green}{HTML}{00A250}
    \definecolor{ansi-green-intense}{HTML}{007427}
    \definecolor{ansi-yellow}{HTML}{DDB62B}
    \definecolor{ansi-yellow-intense}{HTML}{B27D12}
    \definecolor{ansi-blue}{HTML}{208FFB}
    \definecolor{ansi-blue-intense}{HTML}{0065CA}
    \definecolor{ansi-magenta}{HTML}{D160C4}
    \definecolor{ansi-magenta-intense}{HTML}{A03196}
    \definecolor{ansi-cyan}{HTML}{60C6C8}
    \definecolor{ansi-cyan-intense}{HTML}{258F8F}
    \definecolor{ansi-white}{HTML}{C5C1B4}
    \definecolor{ansi-white-intense}{HTML}{A1A6B2}

    % commands and environments needed by pandoc snippets
    % extracted from the output of `pandoc -s`
    \providecommand{\tightlist}{%
      \setlength{\itemsep}{0pt}\setlength{\parskip}{0pt}}
    \DefineVerbatimEnvironment{Highlighting}{Verbatim}{commandchars=\\\{\}}
    % Add ',fontsize=\small' for more characters per line
    \newenvironment{Shaded}{}{}
    \newcommand{\KeywordTok}[1]{\textcolor[rgb]{0.00,0.44,0.13}{\textbf{{#1}}}}
    \newcommand{\DataTypeTok}[1]{\textcolor[rgb]{0.56,0.13,0.00}{{#1}}}
    \newcommand{\DecValTok}[1]{\textcolor[rgb]{0.25,0.63,0.44}{{#1}}}
    \newcommand{\BaseNTok}[1]{\textcolor[rgb]{0.25,0.63,0.44}{{#1}}}
    \newcommand{\FloatTok}[1]{\textcolor[rgb]{0.25,0.63,0.44}{{#1}}}
    \newcommand{\CharTok}[1]{\textcolor[rgb]{0.25,0.44,0.63}{{#1}}}
    \newcommand{\StringTok}[1]{\textcolor[rgb]{0.25,0.44,0.63}{{#1}}}
    \newcommand{\CommentTok}[1]{\textcolor[rgb]{0.38,0.63,0.69}{\textit{{#1}}}}
    \newcommand{\OtherTok}[1]{\textcolor[rgb]{0.00,0.44,0.13}{{#1}}}
    \newcommand{\AlertTok}[1]{\textcolor[rgb]{1.00,0.00,0.00}{\textbf{{#1}}}}
    \newcommand{\FunctionTok}[1]{\textcolor[rgb]{0.02,0.16,0.49}{{#1}}}
    \newcommand{\RegionMarkerTok}[1]{{#1}}
    \newcommand{\ErrorTok}[1]{\textcolor[rgb]{1.00,0.00,0.00}{\textbf{{#1}}}}
    \newcommand{\NormalTok}[1]{{#1}}
    
    % Additional commands for more recent versions of Pandoc
    \newcommand{\ConstantTok}[1]{\textcolor[rgb]{0.53,0.00,0.00}{{#1}}}
    \newcommand{\SpecialCharTok}[1]{\textcolor[rgb]{0.25,0.44,0.63}{{#1}}}
    \newcommand{\VerbatimStringTok}[1]{\textcolor[rgb]{0.25,0.44,0.63}{{#1}}}
    \newcommand{\SpecialStringTok}[1]{\textcolor[rgb]{0.73,0.40,0.53}{{#1}}}
    \newcommand{\ImportTok}[1]{{#1}}
    \newcommand{\DocumentationTok}[1]{\textcolor[rgb]{0.73,0.13,0.13}{\textit{{#1}}}}
    \newcommand{\AnnotationTok}[1]{\textcolor[rgb]{0.38,0.63,0.69}{\textbf{\textit{{#1}}}}}
    \newcommand{\CommentVarTok}[1]{\textcolor[rgb]{0.38,0.63,0.69}{\textbf{\textit{{#1}}}}}
    \newcommand{\VariableTok}[1]{\textcolor[rgb]{0.10,0.09,0.49}{{#1}}}
    \newcommand{\ControlFlowTok}[1]{\textcolor[rgb]{0.00,0.44,0.13}{\textbf{{#1}}}}
    \newcommand{\OperatorTok}[1]{\textcolor[rgb]{0.40,0.40,0.40}{{#1}}}
    \newcommand{\BuiltInTok}[1]{{#1}}
    \newcommand{\ExtensionTok}[1]{{#1}}
    \newcommand{\PreprocessorTok}[1]{\textcolor[rgb]{0.74,0.48,0.00}{{#1}}}
    \newcommand{\AttributeTok}[1]{\textcolor[rgb]{0.49,0.56,0.16}{{#1}}}
    \newcommand{\InformationTok}[1]{\textcolor[rgb]{0.38,0.63,0.69}{\textbf{\textit{{#1}}}}}
    \newcommand{\WarningTok}[1]{\textcolor[rgb]{0.38,0.63,0.69}{\textbf{\textit{{#1}}}}}
    
    
    % Define a nice break command that doesn't care if a line doesn't already
    % exist.
    \def\br{\hspace*{\fill} \\* }
    % Math Jax compatability definitions
    \def\gt{>}
    \def\lt{<}
    % Document parameters
    \title{Chapter10}
    
    
    

    % Pygments definitions
    
\makeatletter
\def\PY@reset{\let\PY@it=\relax \let\PY@bf=\relax%
    \let\PY@ul=\relax \let\PY@tc=\relax%
    \let\PY@bc=\relax \let\PY@ff=\relax}
\def\PY@tok#1{\csname PY@tok@#1\endcsname}
\def\PY@toks#1+{\ifx\relax#1\empty\else%
    \PY@tok{#1}\expandafter\PY@toks\fi}
\def\PY@do#1{\PY@bc{\PY@tc{\PY@ul{%
    \PY@it{\PY@bf{\PY@ff{#1}}}}}}}
\def\PY#1#2{\PY@reset\PY@toks#1+\relax+\PY@do{#2}}

\expandafter\def\csname PY@tok@sx\endcsname{\def\PY@tc##1{\textcolor[rgb]{0.00,0.50,0.00}{##1}}}
\expandafter\def\csname PY@tok@mo\endcsname{\def\PY@tc##1{\textcolor[rgb]{0.40,0.40,0.40}{##1}}}
\expandafter\def\csname PY@tok@c\endcsname{\let\PY@it=\textit\def\PY@tc##1{\textcolor[rgb]{0.25,0.50,0.50}{##1}}}
\expandafter\def\csname PY@tok@err\endcsname{\def\PY@bc##1{\setlength{\fboxsep}{0pt}\fcolorbox[rgb]{1.00,0.00,0.00}{1,1,1}{\strut ##1}}}
\expandafter\def\csname PY@tok@m\endcsname{\def\PY@tc##1{\textcolor[rgb]{0.40,0.40,0.40}{##1}}}
\expandafter\def\csname PY@tok@si\endcsname{\let\PY@bf=\textbf\def\PY@tc##1{\textcolor[rgb]{0.73,0.40,0.53}{##1}}}
\expandafter\def\csname PY@tok@sh\endcsname{\def\PY@tc##1{\textcolor[rgb]{0.73,0.13,0.13}{##1}}}
\expandafter\def\csname PY@tok@no\endcsname{\def\PY@tc##1{\textcolor[rgb]{0.53,0.00,0.00}{##1}}}
\expandafter\def\csname PY@tok@gt\endcsname{\def\PY@tc##1{\textcolor[rgb]{0.00,0.27,0.87}{##1}}}
\expandafter\def\csname PY@tok@sd\endcsname{\let\PY@it=\textit\def\PY@tc##1{\textcolor[rgb]{0.73,0.13,0.13}{##1}}}
\expandafter\def\csname PY@tok@kt\endcsname{\def\PY@tc##1{\textcolor[rgb]{0.69,0.00,0.25}{##1}}}
\expandafter\def\csname PY@tok@nn\endcsname{\let\PY@bf=\textbf\def\PY@tc##1{\textcolor[rgb]{0.00,0.00,1.00}{##1}}}
\expandafter\def\csname PY@tok@mb\endcsname{\def\PY@tc##1{\textcolor[rgb]{0.40,0.40,0.40}{##1}}}
\expandafter\def\csname PY@tok@cm\endcsname{\let\PY@it=\textit\def\PY@tc##1{\textcolor[rgb]{0.25,0.50,0.50}{##1}}}
\expandafter\def\csname PY@tok@nv\endcsname{\def\PY@tc##1{\textcolor[rgb]{0.10,0.09,0.49}{##1}}}
\expandafter\def\csname PY@tok@sb\endcsname{\def\PY@tc##1{\textcolor[rgb]{0.73,0.13,0.13}{##1}}}
\expandafter\def\csname PY@tok@nl\endcsname{\def\PY@tc##1{\textcolor[rgb]{0.63,0.63,0.00}{##1}}}
\expandafter\def\csname PY@tok@vg\endcsname{\def\PY@tc##1{\textcolor[rgb]{0.10,0.09,0.49}{##1}}}
\expandafter\def\csname PY@tok@nb\endcsname{\def\PY@tc##1{\textcolor[rgb]{0.00,0.50,0.00}{##1}}}
\expandafter\def\csname PY@tok@gi\endcsname{\def\PY@tc##1{\textcolor[rgb]{0.00,0.63,0.00}{##1}}}
\expandafter\def\csname PY@tok@bp\endcsname{\def\PY@tc##1{\textcolor[rgb]{0.00,0.50,0.00}{##1}}}
\expandafter\def\csname PY@tok@ss\endcsname{\def\PY@tc##1{\textcolor[rgb]{0.10,0.09,0.49}{##1}}}
\expandafter\def\csname PY@tok@sc\endcsname{\def\PY@tc##1{\textcolor[rgb]{0.73,0.13,0.13}{##1}}}
\expandafter\def\csname PY@tok@nd\endcsname{\def\PY@tc##1{\textcolor[rgb]{0.67,0.13,1.00}{##1}}}
\expandafter\def\csname PY@tok@gd\endcsname{\def\PY@tc##1{\textcolor[rgb]{0.63,0.00,0.00}{##1}}}
\expandafter\def\csname PY@tok@sr\endcsname{\def\PY@tc##1{\textcolor[rgb]{0.73,0.40,0.53}{##1}}}
\expandafter\def\csname PY@tok@cs\endcsname{\let\PY@it=\textit\def\PY@tc##1{\textcolor[rgb]{0.25,0.50,0.50}{##1}}}
\expandafter\def\csname PY@tok@vi\endcsname{\def\PY@tc##1{\textcolor[rgb]{0.10,0.09,0.49}{##1}}}
\expandafter\def\csname PY@tok@kd\endcsname{\let\PY@bf=\textbf\def\PY@tc##1{\textcolor[rgb]{0.00,0.50,0.00}{##1}}}
\expandafter\def\csname PY@tok@o\endcsname{\def\PY@tc##1{\textcolor[rgb]{0.40,0.40,0.40}{##1}}}
\expandafter\def\csname PY@tok@mi\endcsname{\def\PY@tc##1{\textcolor[rgb]{0.40,0.40,0.40}{##1}}}
\expandafter\def\csname PY@tok@nt\endcsname{\let\PY@bf=\textbf\def\PY@tc##1{\textcolor[rgb]{0.00,0.50,0.00}{##1}}}
\expandafter\def\csname PY@tok@c1\endcsname{\let\PY@it=\textit\def\PY@tc##1{\textcolor[rgb]{0.25,0.50,0.50}{##1}}}
\expandafter\def\csname PY@tok@ch\endcsname{\let\PY@it=\textit\def\PY@tc##1{\textcolor[rgb]{0.25,0.50,0.50}{##1}}}
\expandafter\def\csname PY@tok@ow\endcsname{\let\PY@bf=\textbf\def\PY@tc##1{\textcolor[rgb]{0.67,0.13,1.00}{##1}}}
\expandafter\def\csname PY@tok@kp\endcsname{\def\PY@tc##1{\textcolor[rgb]{0.00,0.50,0.00}{##1}}}
\expandafter\def\csname PY@tok@ge\endcsname{\let\PY@it=\textit}
\expandafter\def\csname PY@tok@s\endcsname{\def\PY@tc##1{\textcolor[rgb]{0.73,0.13,0.13}{##1}}}
\expandafter\def\csname PY@tok@se\endcsname{\let\PY@bf=\textbf\def\PY@tc##1{\textcolor[rgb]{0.73,0.40,0.13}{##1}}}
\expandafter\def\csname PY@tok@mf\endcsname{\def\PY@tc##1{\textcolor[rgb]{0.40,0.40,0.40}{##1}}}
\expandafter\def\csname PY@tok@gs\endcsname{\let\PY@bf=\textbf}
\expandafter\def\csname PY@tok@kn\endcsname{\let\PY@bf=\textbf\def\PY@tc##1{\textcolor[rgb]{0.00,0.50,0.00}{##1}}}
\expandafter\def\csname PY@tok@gr\endcsname{\def\PY@tc##1{\textcolor[rgb]{1.00,0.00,0.00}{##1}}}
\expandafter\def\csname PY@tok@go\endcsname{\def\PY@tc##1{\textcolor[rgb]{0.53,0.53,0.53}{##1}}}
\expandafter\def\csname PY@tok@gh\endcsname{\let\PY@bf=\textbf\def\PY@tc##1{\textcolor[rgb]{0.00,0.00,0.50}{##1}}}
\expandafter\def\csname PY@tok@gp\endcsname{\let\PY@bf=\textbf\def\PY@tc##1{\textcolor[rgb]{0.00,0.00,0.50}{##1}}}
\expandafter\def\csname PY@tok@ne\endcsname{\let\PY@bf=\textbf\def\PY@tc##1{\textcolor[rgb]{0.82,0.25,0.23}{##1}}}
\expandafter\def\csname PY@tok@kc\endcsname{\let\PY@bf=\textbf\def\PY@tc##1{\textcolor[rgb]{0.00,0.50,0.00}{##1}}}
\expandafter\def\csname PY@tok@gu\endcsname{\let\PY@bf=\textbf\def\PY@tc##1{\textcolor[rgb]{0.50,0.00,0.50}{##1}}}
\expandafter\def\csname PY@tok@kr\endcsname{\let\PY@bf=\textbf\def\PY@tc##1{\textcolor[rgb]{0.00,0.50,0.00}{##1}}}
\expandafter\def\csname PY@tok@nc\endcsname{\let\PY@bf=\textbf\def\PY@tc##1{\textcolor[rgb]{0.00,0.00,1.00}{##1}}}
\expandafter\def\csname PY@tok@s2\endcsname{\def\PY@tc##1{\textcolor[rgb]{0.73,0.13,0.13}{##1}}}
\expandafter\def\csname PY@tok@s1\endcsname{\def\PY@tc##1{\textcolor[rgb]{0.73,0.13,0.13}{##1}}}
\expandafter\def\csname PY@tok@nf\endcsname{\def\PY@tc##1{\textcolor[rgb]{0.00,0.00,1.00}{##1}}}
\expandafter\def\csname PY@tok@k\endcsname{\let\PY@bf=\textbf\def\PY@tc##1{\textcolor[rgb]{0.00,0.50,0.00}{##1}}}
\expandafter\def\csname PY@tok@vc\endcsname{\def\PY@tc##1{\textcolor[rgb]{0.10,0.09,0.49}{##1}}}
\expandafter\def\csname PY@tok@il\endcsname{\def\PY@tc##1{\textcolor[rgb]{0.40,0.40,0.40}{##1}}}
\expandafter\def\csname PY@tok@cpf\endcsname{\let\PY@it=\textit\def\PY@tc##1{\textcolor[rgb]{0.25,0.50,0.50}{##1}}}
\expandafter\def\csname PY@tok@w\endcsname{\def\PY@tc##1{\textcolor[rgb]{0.73,0.73,0.73}{##1}}}
\expandafter\def\csname PY@tok@na\endcsname{\def\PY@tc##1{\textcolor[rgb]{0.49,0.56,0.16}{##1}}}
\expandafter\def\csname PY@tok@cp\endcsname{\def\PY@tc##1{\textcolor[rgb]{0.74,0.48,0.00}{##1}}}
\expandafter\def\csname PY@tok@ni\endcsname{\let\PY@bf=\textbf\def\PY@tc##1{\textcolor[rgb]{0.60,0.60,0.60}{##1}}}
\expandafter\def\csname PY@tok@mh\endcsname{\def\PY@tc##1{\textcolor[rgb]{0.40,0.40,0.40}{##1}}}

\def\PYZbs{\char`\\}
\def\PYZus{\char`\_}
\def\PYZob{\char`\{}
\def\PYZcb{\char`\}}
\def\PYZca{\char`\^}
\def\PYZam{\char`\&}
\def\PYZlt{\char`\<}
\def\PYZgt{\char`\>}
\def\PYZsh{\char`\#}
\def\PYZpc{\char`\%}
\def\PYZdl{\char`\$}
\def\PYZhy{\char`\-}
\def\PYZsq{\char`\'}
\def\PYZdq{\char`\"}
\def\PYZti{\char`\~}
% for compatibility with earlier versions
\def\PYZat{@}
\def\PYZlb{[}
\def\PYZrb{]}
\makeatother


    % Exact colors from NB
    \definecolor{incolor}{rgb}{0.0, 0.0, 0.5}
    \definecolor{outcolor}{rgb}{0.545, 0.0, 0.0}



    
    % Prevent overflowing lines due to hard-to-break entities
    \sloppy 
    % Setup hyperref package
    \hypersetup{
      breaklinks=true,  % so long urls are correctly broken across lines
      colorlinks=true,
      urlcolor=urlcolor,
      linkcolor=linkcolor,
      citecolor=citecolor,
      }
    % Slightly bigger margins than the latex defaults
    
    \geometry{verbose,tmargin=1in,bmargin=1in,lmargin=1in,rmargin=1in}
    
    

    \begin{document}
    
    
    \maketitle
    
    

    
    \section{10章 ヒープ}\label{ux7ae0ux30d2ux30fcux30d7}

\begin{itemize}
\tightlist
\item
  Yusuke Matsuyama
\item
  7/14@book-seminar2017
\item
  via jupyter RISE
\end{itemize}

    \section{おきもち}\label{ux304aux304dux3082ux3061}

\begin{enumerate}
\def\labelenumi{\arabic{enumi}.}
\tightlist
\item
  優先度付きキューを実装したい
\item
  二分探索木を使えば優先度付きキューは作れる

  \begin{itemize}
  \tightlist
  \item
    しかし非常に難しい
  \end{itemize}
\item
  二分ヒープを使えば比較的簡単

  \begin{itemize}
  \tightlist
  \item
    つくろう!
  \end{itemize}
\end{enumerate}

    \section{アウトライン}\label{ux30a2ux30a6ux30c8ux30e9ux30a4ux30f3}

\begin{enumerate}
\def\labelenumi{\arabic{enumi}.}
\tightlist
\item
  二分ヒープの説明

  \begin{enumerate}
  \def\labelenumii{\arabic{enumii}.}
  \tightlist
  \item
    完全二分木
  \item
    二分ヒープ

    \begin{enumerate}
    \def\labelenumiii{\arabic{enumiii}.}
    \tightlist
    \item
      max-ヒープ
    \end{enumerate}
  \end{enumerate}
\item
  max-ヒープの実装
\item
  優先度付きキューの実装
\end{enumerate}

    \section{完全二分木}\label{ux5b8cux5168ux4e8cux5206ux6728}

\begin{itemize}
\tightlist
\item
  すべての葉が同じ深さを持ち、全ての内部接点の子の数が2であるような二分木(a)
\item
  深さの差がmax1でも、ちゃんと左側から埋まってるような木も(おおよそ(?))二分木(b)
\end{itemize}

\begin{figure}
\centering
\includegraphics{./imgs/CBT.png}
\caption{}
\end{figure}

    \section{二分ヒープ}\label{ux4e8cux5206ux30d2ux30fcux30d7}

完全二分木の各ノードに割り当てられたキーが、配列の各要素に対応したデータ構造
\includegraphics{./imgs/BH.png}

    \section{indexの計算}\label{indexux306eux8a08ux7b97}

添字iが与えられた時:

    \begin{Verbatim}[commandchars=\\\{\}]
{\color{incolor}In [{\color{incolor}1}]:} \PY{k+kn}{import} \PY{n+nn}{math}
        \PY{n}{L} \PY{o}{=} \PY{l+m+mi}{10} \PY{c+c1}{\PYZsh{} length of heap}
        
        \PY{c+c1}{\PYZsh{} origin 1}
        
        \PY{n}{parent}        \PY{o}{=} \PY{k}{lambda}  \PY{n}{i} \PY{p}{:} \PY{n}{math}\PY{o}{.}\PY{n}{floor}\PY{p}{(}\PY{n}{i}\PY{o}{/}\PY{l+m+mi}{2}\PY{p}{)}
        \PY{n}{left\PYZus{}child}   \PY{o}{=} \PY{k}{lambda}  \PY{n}{i} \PY{p}{:} \PY{l+m+mi}{2} \PY{o}{*} \PY{n}{i} 
        \PY{n}{right\PYZus{}child} \PY{o}{=} \PY{k}{lambda}  \PY{n}{i} \PY{p}{:} \PY{l+m+mi}{2} \PY{o}{*} \PY{n}{i} \PY{o}{+} \PY{l+m+mi}{1}
        \PY{n}{parent}\PY{p}{(}\PY{l+m+mi}{5}\PY{p}{)}\PY{p}{,}\PY{n}{left\PYZus{}child}\PY{p}{(}\PY{l+m+mi}{5}\PY{p}{)}\PY{p}{,} \PY{n}{right\PYZus{}child}\PY{p}{(}\PY{l+m+mi}{5}\PY{p}{)}
\end{Verbatim}

            \begin{Verbatim}[commandchars=\\\{\}]
{\color{outcolor}Out[{\color{outcolor}1}]:} (2, 10, 11)
\end{Verbatim}
        
    \section{max-ヒープ条件}\label{max-ux30d2ux30fcux30d7ux6761ux4ef6}

\begin{itemize}
\tightlist
\item
  節点のキーがその親のキー以下であること(逆: minヒープ条件)
\item
  制約があるのは親子間のみ(兄弟間にはない)
\end{itemize}

\subsection{max-ヒープ}\label{max-ux30d2ux30fcux30d7}

\begin{itemize}
\tightlist
\item
  max ヒープ条件を満たす二分ヒープ \includegraphics{./imgs/BH.png}
\end{itemize}

    \section{ヒープへの要素追加}\label{ux30d2ux30fcux30d7ux3078ux306eux8981ux7d20ux8ffdux52a0}

\subsection{maxheapfi(A,i)}\label{maxheapfiai}

A{[}i{]}をmaxヒープ条件を満たすまで葉に向かって下降

\begin{enumerate}
\def\labelenumi{\arabic{enumi}.}
\tightlist
\item
  if max(左子のキー、右子のキー) \textgreater{} 自分のキー then
  そいつと自分を入れ替える
\item
  1.を繰り返す
\end{enumerate}

\subsection{max-ヒープの構築}\label{max-ux30d2ux30fcux30d7ux306eux69cbux7bc9}

子を持つ節点の中で添字が最大の節点sから逆順にmaxheapfi(A,s)をすればいい

\begin{figure}
\centering
\includegraphics{imgs/maxheapfi.png}
\caption{maxheapfi.png}
\end{figure}

    \section{ヒープ構築の計算量}\label{ux30d2ux30fcux30d7ux69cbux7bc9ux306eux8a08ux7b97ux91cf}

要素数をHとする。

\subsection{maxheapfi :}\label{maxheapfi}

\begin{verbatim}
O(log 木の高さ)
\end{verbatim}

\subsection{ヒープの構築:}\label{ux30d2ux30fcux30d7ux306eux69cbux7bc9}

\begin{verbatim}
- 高さ1の部分木 H/2個に対してmaxheapfi
- 高さ2の部分木 H/4個に対してmaxheapfi
- ... 高さlog Hの部分木1つ に対してmaxheapfi
\end{verbatim}

\[ H * \sum^{logH}_{k=1} \frac{k}{2^k} = O(H) \]

    \section{問題1(15分)
最大ヒープの実装(ALDS1\_9\_B)}\label{ux554fux984c115ux5206-ux6700ux5927ux30d2ux30fcux30d7ux306eux5b9fux88c5alds1_9_b}

与えられた配列から、max-ヒープを実装してください

\subsection{入力}\label{ux5165ux529b}

\begin{itemize}
\tightlist
\item
  最初の行:配列サイズH
\item
  H個の配列要素
\end{itemize}

\begin{verbatim}
10
4 1 3 2 16 9 10 14 8 7 
\end{verbatim}

\subsection{出力例}\label{ux51faux529bux4f8b}

max-ヒープの節点の値を1\ldots{}Hに向かって順番に空白区切で出力

\begin{verbatim}
16 14 10 8 7 9 3 2 4 1
\end{verbatim}

\subsection{制約}\label{ux5236ux7d04}

\[ 1 <= H <= 500000\] \[ -2,000,000,000 <= 節点の値 <= 2,000,000,000 \]

    \begin{Verbatim}[commandchars=\\\{\}]
{\color{incolor}In [{\color{incolor}8}]:} \PY{o}{!} cat ./test.sh
\end{Verbatim}

    \begin{Verbatim}[commandchars=\\\{\}]
\#!/bin/bash

code=\$1

set -eu

[ "\$code" == "" ] \&\& code="template.c"

\# This is setting for C users
\#compile="gcc -Wall \$code"
\#exec\_command="./a.out"

\# for example python users
compile=""
exec\_command="python \$code"

test\_sets="A1 B1 B2"

temp=\$(mktemp aojtest.XXXXXX)
trap "rm \$temp" EXIT

\$compile
for case in \$test\_sets; do
  echo [case \$case]
  \$exec\_command < input/\$\{case\}.txt > \$temp
  if diff -u output/\$\{case\}.txt \$temp; then
    echo OK
  fi
done

    \end{Verbatim}

    \begin{Verbatim}[commandchars=\\\{\}]
{\color{incolor}In [{\color{incolor}2}]:} \PY{c+c1}{\PYZsh{} 解答1}
        \PY{k+kn}{from}  \PY{n+nn}{math} \PY{k}{import} \PY{n}{floor}
        \PY{n}{H} \PY{o}{=} \PY{l+m+mi}{0}
        \PY{n}{A} \PY{o}{=} \PY{p}{[}\PY{p}{]}
        
        \PY{k}{def} \PY{n+nf}{maxheapfi}\PY{p}{(}\PY{n}{i}\PY{p}{,}\PY{n}{A}\PY{p}{,}\PY{n}{H}\PY{p}{)}\PY{p}{:}
                \PY{n}{l} \PY{o}{=} \PY{l+m+mi}{2} \PY{o}{*} \PY{p}{(}\PY{n}{i}  \PY{o}{+} \PY{l+m+mi}{1}\PY{p}{)} \PY{o}{\PYZhy{}} \PY{l+m+mi}{1} 
                \PY{n}{r} \PY{o}{=} \PY{l+m+mi}{2} \PY{o}{*} \PY{p}{(}\PY{n}{i}  \PY{o}{+} \PY{l+m+mi}{1}\PY{p}{)} 
                
                \PY{k}{if} \PY{n}{l}  \PY{o}{\PYZlt{}} \PY{n}{H} \PY{o+ow}{and} \PY{n}{A}\PY{p}{[}\PY{n}{l}\PY{p}{]} \PY{o}{\PYZgt{}} \PY{n}{A}\PY{p}{[}\PY{n}{i}\PY{p}{]} \PY{p}{:}
                    \PY{n}{largest} \PY{o}{=} \PY{n}{l}
                \PY{k}{else}\PY{p}{:}
                    \PY{n}{largest} \PY{o}{=} \PY{n}{i}
                
                \PY{k}{if} \PY{n}{r} \PY{o}{\PYZlt{}} \PY{n}{H} \PY{o+ow}{and} \PY{n}{A}\PY{p}{[}\PY{n}{r}\PY{p}{]} \PY{o}{\PYZgt{}} \PY{n}{A}\PY{p}{[}\PY{n}{largest}\PY{p}{]} \PY{p}{:}
                    \PY{n}{largest} \PY{o}{=} \PY{n}{r}
                
                \PY{k}{if} \PY{n}{largest} \PY{o}{!=} \PY{n}{i}\PY{p}{:}
                    \PY{n}{tmp} \PY{o}{=} \PY{n}{A}\PY{p}{[}\PY{n}{i}\PY{p}{]}
                    \PY{n}{A}\PY{p}{[}\PY{n}{i}\PY{p}{]} \PY{o}{=} \PY{n}{A}\PY{p}{[}\PY{n}{largest}\PY{p}{]}
                    \PY{n}{A}\PY{p}{[}\PY{n}{largest}\PY{p}{]} \PY{o}{=} \PY{n}{tmp}
                    \PY{n}{maxheapfi}\PY{p}{(}\PY{n}{largest}\PY{p}{,}\PY{n}{A}\PY{p}{,}\PY{n}{H}\PY{p}{)}
        
        \PY{k}{if} \PY{n}{\PYZus{}\PYZus{}name\PYZus{}\PYZus{}}\PY{o}{==} \PY{l+s+s1}{\PYZsq{}}\PY{l+s+s1}{\PYZus{}\PYZus{}main\PYZus{}\PYZus{}}\PY{l+s+s1}{\PYZsq{}}\PY{p}{:}
            \PY{c+c1}{\PYZsh{}H = int(input().strip())}
            \PY{c+c1}{\PYZsh{}A = map(int, input().strip().split(\PYZsq{} \PYZsq{}))}
            \PY{n}{lines} \PY{o}{=} \PY{p}{[}\PY{n}{x}\PY{o}{.}\PY{n}{strip}\PY{p}{(}\PY{p}{)} \PY{k}{for} \PY{n}{x} \PY{o+ow}{in} \PY{n+nb}{open}\PY{p}{(}\PY{l+s+s1}{\PYZsq{}}\PY{l+s+s1}{./input/A1.txt}\PY{l+s+s1}{\PYZsq{}}\PY{p}{)}\PY{o}{.}\PY{n}{readlines}\PY{p}{(}\PY{p}{)}\PY{p}{]}
            \PY{n}{H} \PY{o}{=} \PY{n+nb}{int}\PY{p}{(}\PY{n}{lines}\PY{p}{[}\PY{l+m+mi}{0}\PY{p}{]}\PY{p}{)}
            \PY{n}{A} \PY{o}{=} \PY{n+nb}{list}\PY{p}{(}\PY{n+nb}{map}\PY{p}{(}\PY{n+nb}{int}\PY{p}{,} \PY{n}{lines}\PY{p}{[}\PY{l+m+mi}{1}\PY{p}{]}\PY{o}{.}\PY{n}{split}\PY{p}{(}\PY{l+s+s1}{\PYZsq{}}\PY{l+s+s1}{ }\PY{l+s+s1}{\PYZsq{}}\PY{p}{)}\PY{p}{)}\PY{p}{)}
            
            \PY{k}{for} \PY{n}{k} \PY{o+ow}{in} \PY{n+nb}{range}\PY{p}{(}\PY{n}{floor}\PY{p}{(}\PY{n}{H}\PY{o}{/}\PY{l+m+mi}{2}\PY{p}{)}\PY{o}{\PYZhy{}}\PY{l+m+mi}{1}\PY{p}{,}\PY{o}{\PYZhy{}}\PY{l+m+mi}{1}\PY{p}{,}\PY{o}{\PYZhy{}}\PY{l+m+mi}{1}\PY{p}{)}\PY{p}{:}
                \PY{n}{maxheapfi}\PY{p}{(}\PY{n}{k}\PY{p}{,}\PY{n}{A}\PY{p}{,}\PY{n}{H}\PY{p}{)}
                
            \PY{n+nb}{print}\PY{p}{(}\PY{n}{A}\PY{p}{)} 
                
            
            
\end{Verbatim}

    \begin{Verbatim}[commandchars=\\\{\}]
[16, 14, 10, 8, 7, 9, 3, 2, 4, 1]

    \end{Verbatim}

    \section{優先度付きキュー}\label{ux512aux5148ux5ea6ux4ed8ux304dux30adux30e5ux30fc}

キーを持つ要素 の集合 \#\# insert(S,k): 集合Sに要素kを挿入

\subsection{extractMax(S):}\label{extractmaxs}

最大のキーを持つSの要素をSから削除し、その値を返す

    \section{要素の追加}\label{ux8981ux7d20ux306eux8ffdux52a0}

\begin{enumerate}
\def\labelenumi{\arabic{enumi}.}
\tightlist
\item
  ヒープの末尾に追加
\item
  ヒープ条件の確保

  \begin{itemize}
  \tightlist
  \item
    while(親より強い) : 親と入れかわる
    \includegraphics{./imgs/insertPQ.png}
  \end{itemize}
\end{enumerate}

    \section{要素の取り出し}\label{ux8981ux7d20ux306eux53d6ux308aux51faux3057}

\begin{enumerate}
\def\labelenumi{\arabic{enumi}.}
\tightlist
\item
  根を記録
\item
  ヒープの一番末尾の要素を根に移動
\item
  ヒープサイズH--
\item
  根からmaxheapfiを実行
\item
  記憶していた旧根の値をreturn \includegraphics{imgs/popPQ.png}
\end{enumerate}

    \section{問題2 :
優先度付きキューの実装(ALDS1\_9\_C)}\label{ux554fux984c2-ux512aux5148ux5ea6ux4ed8ux304dux30adux30e5ux30fcux306eux5b9fux88c5alds1_9_c}

\subsection{入力:}\label{ux5165ux529b}

優先度付きキューへの複数の命令 - insert k - extract - end

\subsection{出力:}\label{ux51faux529b}

extract命令ごとに、優先度付きキューから取り出される値を一行ごとに出力

\subsection{制約}\label{ux5236ux7d04}

\[命令の数 <= 2,000,000\] \[0 <= k <= 2,000,000,000\]

    - 入力

\begin{verbatim}
insert 8
insert 2
extract
insert 10
extract
imsert 11
extract
extract
\end{verbatim}

\begin{itemize}
\item
  出力

\begin{verbatim}
8
10
11
2
\end{verbatim}
\end{itemize}

    \begin{Verbatim}[commandchars=\\\{\}]
{\color{incolor}In [{\color{incolor}5}]:} \PY{k+kn}{from} \PY{n+nn}{math} \PY{k}{import} \PY{n}{floor}
        
        \PY{k}{def} \PY{n+nf}{maxheapfi}\PY{p}{(}\PY{n}{i}\PY{p}{)}\PY{p}{:}
            \PY{k}{global} \PY{n}{H}
            \PY{k}{global} \PY{n}{A}
            \PY{n}{l} \PY{o}{=} \PY{l+m+mi}{2} \PY{o}{*} \PY{p}{(}\PY{n}{i}\PY{p}{)} \PY{o}{+} \PY{l+m+mi}{1}  
            \PY{n}{r} \PY{o}{=} \PY{l+m+mi}{2} \PY{o}{*} \PY{p}{(}\PY{n}{i} \PY{o}{+}\PY{l+m+mi}{1}\PY{p}{)}
        
            \PY{k}{if} \PY{n}{l}  \PY{o}{\PYZlt{}} \PY{n}{H} \PY{o+ow}{and} \PY{n}{A}\PY{p}{[}\PY{n}{l}\PY{p}{]} \PY{o}{\PYZgt{}} \PY{n}{A}\PY{p}{[}\PY{n}{i}\PY{p}{]} \PY{p}{:}
                \PY{n}{largest} \PY{o}{=} \PY{n}{l}
            \PY{k}{else}\PY{p}{:}
                \PY{n}{largest} \PY{o}{=} \PY{n}{i}
        
            \PY{k}{if} \PY{n}{r} \PY{o}{\PYZlt{}} \PY{n}{H} \PY{o+ow}{and} \PY{n}{A}\PY{p}{[}\PY{n}{r}\PY{p}{]} \PY{o}{\PYZgt{}} \PY{n}{A}\PY{p}{[}\PY{n}{largest}\PY{p}{]} \PY{p}{:}
                \PY{n}{largest} \PY{o}{=} \PY{n}{r}
            \PY{k}{if} \PY{n}{largest} \PY{o}{!=} \PY{n}{i}\PY{p}{:}
                \PY{n}{tmp} \PY{o}{=} \PY{n}{A}\PY{p}{[}\PY{n}{i}\PY{p}{]}
                \PY{n}{A}\PY{p}{[}\PY{n}{i}\PY{p}{]} \PY{o}{=} \PY{n}{A}\PY{p}{[}\PY{n}{largest}\PY{p}{]}
                \PY{n}{A}\PY{p}{[}\PY{n}{largest}\PY{p}{]} \PY{o}{=} \PY{n}{tmp}
                \PY{n}{maxheapfi}\PY{p}{(}\PY{n}{largest}\PY{p}{)}
            
        \PY{k}{def} \PY{n+nf}{extract}\PY{p}{(}\PY{p}{)}\PY{p}{:}
            \PY{k}{global} \PY{n}{H}
            \PY{k}{global} \PY{n}{A}
            \PY{k}{if} \PY{n}{H} \PY{o}{\PYZlt{}} \PY{l+m+mi}{0} \PY{p}{:} \PY{k}{return} \PY{o}{\PYZhy{}}\PY{l+m+mi}{1000000}
            \PY{n}{maxv} \PY{o}{=} \PY{n}{A}\PY{p}{[}\PY{l+m+mi}{0}\PY{p}{]}
            \PY{n}{A}\PY{p}{[}\PY{l+m+mi}{0}\PY{p}{]} \PY{o}{=} \PY{n}{A}\PY{p}{[}\PY{n}{H}\PY{o}{\PYZhy{}}\PY{l+m+mi}{1}\PY{p}{]}
            \PY{n}{H} \PY{o}{\PYZhy{}}\PY{o}{=} \PY{l+m+mi}{1}
            \PY{n}{maxheapfi}\PY{p}{(}\PY{l+m+mi}{0}\PY{p}{)}
            \PY{k}{return} \PY{n}{maxv}
        
        \PY{k}{def} \PY{n+nf}{increasekey}\PY{p}{(}\PY{n}{i}\PY{p}{,} \PY{n}{key}\PY{p}{)}\PY{p}{:}
            \PY{k}{global} \PY{n}{H}
            \PY{k}{global} \PY{n}{A}
            \PY{k}{if} \PY{n}{key} \PY{o}{\PYZlt{}} \PY{n}{A}\PY{p}{[}\PY{n}{i}\PY{p}{]}\PY{p}{:}
                \PY{k}{return}
            \PY{n}{A}\PY{p}{[}\PY{n}{i}\PY{p}{]} \PY{o}{=} \PY{n}{key}
            \PY{k}{while}\PY{p}{(} \PY{n}{i} \PY{o}{\PYZgt{}} \PY{l+m+mi}{0} \PY{o+ow}{and} \PY{n}{A}\PY{p}{[}\PY{n}{floor}\PY{p}{(}\PY{p}{(}\PY{n}{i}\PY{o}{\PYZhy{}}\PY{l+m+mi}{1}\PY{p}{)}\PY{o}{/}\PY{l+m+mi}{2}\PY{p}{)}\PY{p}{]} \PY{o}{\PYZlt{}} \PY{n}{A}\PY{p}{[}\PY{n}{i}\PY{p}{]}\PY{p}{)}\PY{p}{:}
                \PY{n}{tmp} \PY{o}{=} \PY{n}{A}\PY{p}{[}\PY{n}{i}\PY{p}{]}
                \PY{n}{A}\PY{p}{[}\PY{n}{i}\PY{p}{]} \PY{o}{=} \PY{n}{A}\PY{p}{[}\PY{n}{floor}\PY{p}{(}\PY{p}{(}\PY{n}{i}\PY{o}{\PYZhy{}}\PY{l+m+mi}{1}\PY{p}{)}\PY{o}{/}\PY{l+m+mi}{2}\PY{p}{)} \PY{p}{]} 
                \PY{n}{A}\PY{p}{[}\PY{n}{floor}\PY{p}{(}\PY{p}{(}\PY{n}{i}\PY{o}{\PYZhy{}}\PY{l+m+mi}{1}\PY{p}{)}\PY{o}{/}\PY{l+m+mi}{2}\PY{p}{)}\PY{p}{]}  \PY{o}{=} \PY{n}{tmp}
                \PY{n}{i} \PY{o}{=} \PY{n}{floor}\PY{p}{(}\PY{p}{(}\PY{n}{i}\PY{o}{\PYZhy{}}\PY{l+m+mi}{1}\PY{p}{)}\PY{o}{/}\PY{l+m+mi}{2}\PY{p}{)}
            
                
        \PY{k}{def} \PY{n+nf}{insert}\PY{p}{(}\PY{n}{key}\PY{p}{)}\PY{p}{:}
            \PY{k}{global} \PY{n}{H}
            \PY{k}{global} \PY{n}{A}
            \PY{n}{H} \PY{o}{+}\PY{o}{=} \PY{l+m+mi}{1}
            \PY{k}{if} \PY{n+nb}{len}\PY{p}{(}\PY{n}{A}\PY{p}{)} \PY{o}{\PYZlt{}} \PY{n}{H}\PY{p}{:}
                \PY{n}{A}\PY{o}{.}\PY{n}{append}\PY{p}{(}\PY{o}{\PYZhy{}}\PY{l+m+mi}{1000000}\PY{p}{)}
            \PY{k}{else}\PY{p}{:}
                \PY{n}{A}\PY{p}{[}\PY{n}{H}\PY{o}{\PYZhy{}}\PY{l+m+mi}{1}\PY{p}{]} \PY{o}{=} \PY{o}{\PYZhy{}}\PY{l+m+mi}{1000000}
            \PY{n}{increasekey}\PY{p}{(}\PY{n}{H}\PY{o}{\PYZhy{}}\PY{l+m+mi}{1}\PY{p}{,}\PY{n}{key}\PY{p}{)}
        
        \PY{k}{def} \PY{n+nf}{main}\PY{p}{(}\PY{p}{)}\PY{p}{:}
            \PY{k}{global} \PY{n}{H}
            \PY{k}{global} \PY{n}{A}
            \PY{n}{H} \PY{o}{=} \PY{l+m+mi}{0}
            \PY{n}{A} \PY{o}{=} \PY{p}{[}\PY{p}{]}
            \PY{n}{opr} \PY{o}{=} \PY{k+kc}{None}
            \PY{n}{count} \PY{o}{=} \PY{l+m+mi}{0}
            \PY{n}{oprators} \PY{o}{=} \PY{p}{[}\PY{n}{x}\PY{o}{.}\PY{n}{strip}\PY{p}{(}\PY{p}{)}\PY{o}{.}\PY{n}{split}\PY{p}{(}\PY{l+s+s1}{\PYZsq{}}\PY{l+s+s1}{ }\PY{l+s+s1}{\PYZsq{}}\PY{p}{)} \PY{k}{for} \PY{n}{x} \PY{o+ow}{in} \PY{n+nb}{open}\PY{p}{(}\PY{l+s+s1}{\PYZsq{}}\PY{l+s+s1}{./input/B2.txt}\PY{l+s+s1}{\PYZsq{}}\PY{p}{)}\PY{o}{.}\PY{n}{readlines}\PY{p}{(}\PY{p}{)}\PY{p}{]}
            \PY{k}{for} \PY{n}{opr} \PY{o+ow}{in} \PY{n}{oprators}\PY{p}{:}
            \PY{c+c1}{\PYZsh{}while(True):}
                \PY{c+c1}{\PYZsh{}opr = input().strip().split(\PYZsq{} \PYZsq{})}
                \PY{k}{if} \PY{n}{opr}\PY{p}{[}\PY{l+m+mi}{0}\PY{p}{]} \PY{o}{==} \PY{l+s+s1}{\PYZsq{}}\PY{l+s+s1}{insert}\PY{l+s+s1}{\PYZsq{}}\PY{p}{:}
                    \PY{n}{insert}\PY{p}{(}\PY{n+nb}{int}\PY{p}{(}\PY{n}{opr}\PY{p}{[}\PY{l+m+mi}{1}\PY{p}{]}\PY{p}{)}\PY{p}{)}
                \PY{k}{if} \PY{n}{opr}\PY{p}{[}\PY{l+m+mi}{0}\PY{p}{]} \PY{o}{==} \PY{l+s+s1}{\PYZsq{}}\PY{l+s+s1}{extract}\PY{l+s+s1}{\PYZsq{}}\PY{p}{:}
                    \PY{n+nb}{print}\PY{p}{(}\PY{n}{extract}\PY{p}{(}\PY{p}{)}\PY{p}{)}
                \PY{k}{if} \PY{n}{opr}\PY{p}{[}\PY{l+m+mi}{0}\PY{p}{]} \PY{o}{==} \PY{l+s+s1}{\PYZsq{}}\PY{l+s+s1}{end}\PY{l+s+s1}{\PYZsq{}}\PY{p}{:}
                    \PY{k}{break}
                
        \PY{n}{main}\PY{p}{(}\PY{p}{)}
\end{Verbatim}

    \begin{Verbatim}[commandchars=\\\{\}]
999999999
120
100
777
100
100
100
100
120
88

    \end{Verbatim}

    \begin{Verbatim}[commandchars=\\\{\}]
{\color{incolor}In [{\color{incolor}6}]:} \PY{n+nb}{sorted}\PY{p}{(}\PY{p}{[}\PY{l+m+mi}{1}\PY{p}{,}\PY{l+m+mi}{4}\PY{p}{,}\PY{l+m+mi}{2}\PY{p}{,}\PY{l+m+mi}{4}\PY{p}{,}\PY{l+m+mi}{3}\PY{p}{]}\PY{p}{)}
\end{Verbatim}

            \begin{Verbatim}[commandchars=\\\{\}]
{\color{outcolor}Out[{\color{outcolor}6}]:} [1, 2, 3, 4, 4]
\end{Verbatim}
        
    \begin{Verbatim}[commandchars=\\\{\}]
{\color{incolor}In [{\color{incolor} }]:} 
\end{Verbatim}

    \begin{Verbatim}[commandchars=\\\{\}]
{\color{incolor}In [{\color{incolor} }]:} 
\end{Verbatim}

    \begin{Verbatim}[commandchars=\\\{\}]
{\color{incolor}In [{\color{incolor}7}]:} \PY{n}{foo} \PY{o}{=} \PY{l+m+mi}{4}
        \PY{k}{def} \PY{n+nf}{hoge}\PY{p}{(}\PY{n}{x}\PY{p}{)}\PY{p}{:}
            \PY{n}{hoge} \PY{o}{+}\PY{o}{=} \PY{l+m+mi}{1}
            \PY{k}{return} \PY{n}{hoge}
        
        \PY{n}{hoge}\PY{p}{(}\PY{l+m+mi}{3}\PY{p}{)}
\end{Verbatim}

    \begin{Verbatim}[commandchars=\\\{\}]

        ---------------------------------------------------------------------------

        UnboundLocalError                         Traceback (most recent call last)

        <ipython-input-7-38dc3c6c948e> in <module>()
          4     return hoge
          5 
    ----> 6 hoge(3)
    

        <ipython-input-7-38dc3c6c948e> in hoge(x)
          1 foo = 4
          2 def hoge(x):
    ----> 3     hoge += 1
          4     return hoge
          5 


        UnboundLocalError: local variable 'hoge' referenced before assignment

    \end{Verbatim}

    \begin{Verbatim}[commandchars=\\\{\}]
{\color{incolor}In [{\color{incolor} }]:} 
\end{Verbatim}


    % Add a bibliography block to the postdoc
    
    
    
    \end{document}
